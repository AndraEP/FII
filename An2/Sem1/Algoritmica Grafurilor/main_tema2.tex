\documentclass[12pt]{article}
\usepackage[margin=1in]{geometry} 
\usepackage{amsmath,amsthm,amssymb,amsfonts,listings}
 
\newcommand{\N}{\mathbb{N}}
\newcommand{\Z}{\mathbb{Z}}
 
\newenvironment{rcases}
 {\left.\begin{aligned}}
 	{\end{aligned}\right\rbrace}
 
\newenvironment{problema}[2][Problema]{\begin{trivlist}
\item[\hskip \labelsep {\bfseries #1}\hskip \labelsep {\bfseries #2.}]}{\end{trivlist}}

\begin{document}
 
\title{\textbf{Tema 2, Algoritmica Grafurilor}}
\author{
  Bujoreanu Vlad, Paduraru Andra - Elena\\
  \text{Grupa A4, Anul 2}
}
\date{}
\maketitle
 
\begin{problema}{1}
\end{problema}

\begin{proof}[Solutie]

Fie G = (V, E), C - cuplaj perfect in G.
\newline
Algoritm:
\begin{lstlisting}[mathescape]

build_tree(G, C)
{
    // A si B vor fi multimi stabile de cardinal maxim in T
    A = {};
    B = {};
    E' = {};
    viz[] = {0};
    queue q;
    a = random(1, |V|);
    q.push(a);
    viz[a] = 1;
    A = A $\cup$ {a};
    b = perechea lui a din cuplajul perfect C;
    B = B $\cup$ {b};
    E' = E' $\cup$ {ab};
    while (q.size() > 0)
    {
    	x = q.front(); q.pop();
    	foreach t = xy, x $\in$ E(G) && viz[y] == 0
    	{
    		$\exists$ t' = yz a.i. t' $\in$ C
    		viz[y] = viz[z] = 1;
    		q.push(y); q.push(z);
    		if (x $\in$ A)
    		{
    			B = B $\cup$ {y};
    			A = A $\cup$ {z};
    		}
    		else
    		{
    			A = A $\cup$ {y};
    			B = B $\cup$ {z};
    		}
    		E' = E' $\cup$ {xy, yz};
    	}
    	V' = A $\cup$ B;
    }
    return T = (V', E');
}
\end{lstlisting}

Corectitudine:
\newline
Algoritmul simuleaza o parcurgere in adancime (BFS) de complexitate $\mathcal{O}$($\vert$V$\vert$+$\vert$E$\vert$).
\newline
Fiecare nod este evaluat o singura data datorita vectorului boolean viz.
\newline
Pentru fiecare nod putem gasi o pereche in cuplaj, deoarece cuplajul este perfect.
\newline
Fiecare muchie are un capat in A si altul in B, astfel ca multimile A si B formeaza bipartitia arborelui T rezultat.
\newline
G admite arbore partial $\rightarrow$ este conex $\rightarrow$ fiecare nod este parcurs cel putin o data $\rightarrow$ A si B au cardinal maxim.

\end{proof}

\newpage

\begin{problema}{2}
\end{problema}
 
\begin{proof}[Solutie]: \newline

"$\implies$":
\newline
Graful G admite un cuplaj de ordin p $\implies$ $\vert$G$\vert$-2p noduri nu fac parte din cuplaj. Graful G-S poate contine maxim $\vert$G$\vert$-2p care nu fac parte din cuplajul de ordin p, deci $\forall$ q(G-S), cate un nod cu exceptia a $\vert$G$\vert$-2p vor avea o extremitate a muchiei in S. Intrucat extremitatile sunt diferite, $\implies$ q(G-S) - ($\vert$G$\vert$-2p) $\leq$ $\vert$S$\vert$ $\implies$ q(G-S) $\leq$ $\vert$S$\vert$ + $\vert$G$\vert$ - 2p, $\forall$ S $\subseteq$ G.
\newline\newline
"$\impliedby$":
\newline
Presupunem $\forall$G' cu $\vert$G'$\vert$$\leq$$\vert$G$\vert$, daca $\exists$p' a.i. q(G'-S') $\leq$ $\vert$S'$\vert$ + $\vert$G'$\vert$ - 2p', $\forall$ S' $\subseteq$ G' $\implies$ cardinalul maxim al unui cuplaj e p'
\newline
Demonstratie prin inductie:
\newline
P(1), P(2): adevarat
\newline
P(G): q(G-S) $\leq$ $\vert$S$\vert$ + $\vert$G$\vert$ - 2p, $\forall$ S $\subseteq$ G
\newline
Consideram $S_0$ a.i. q(G-$S_0$) = $\vert$$S_0$$\vert$ + $\vert$G$\vert$ - 2p
\newline
Ipoteza inductiva este respectata $\forall$ componenta conexa impara din G-$S_0$.
\newline
q(G-$S_0$) = $\vert$$S_0$$\vert$ + $\vert$G$\vert$ - 2p $\implies$ $\vert$$S_0$$\vert$ = q(G-$S_0$) - $\vert$G$\vert$ + 2p. Avem $\vert$G$\vert$ + 2p noduri ramase neacoperite de cuplaj, deoarece fiecare componenta conexa impara are exact un nod neacoperit, iar acesta poate fi imperecheat cu un nod din $S_0$ $\implies$ avem cuplaj de ordin p.
\newline
p este maximal cu aceasta proprietate din implicatia directa $\implies$ cuplajul gasit (de ordin p) este maximal.

\end{proof}

\newpage

\begin{problema}{3}
\end{problema}
 
\begin{proof}[Solutie]:\newline

a. \newline
"$\implies$" \newline
Prin eliminarea unei muchii de pe fiecare drum de la u la v, obtinem cel putin doua componente conexe, dintre care intr-una se gaseste u, iar in cealalta se gaseste v.
\newline
e $\in$ E(G) - muchie aleasa arbitrar.
\newline
Daca $\exists$$T^*$ APM pentru G a.i. e $\in$ E($T^*$) $\implies$ c($T^*$) = $min_{T\in T_0 }$(c(T))
\newline
Printr-o taietura S, se genereaza in graf doua componente conexe $G_u$ si $G_v$, cu $T_u^*$ si $T_v^*$, APM ai celor doua componente conexe.
\newline
\[
\begin{rcases}
\begin{gathered}
c(T^*) = c(T_u^*) + c(T_v^*) + min(S)\\
e \in E(T^*)
\end{gathered} \end{rcases} \Rightarrow \begin{gathered}
e = min(S)
\end{gathered}
\]

$\exists T^* APM, e \in E(T^*) \implies e = min(S), S\ taietura\ in\ G$
\newline\newline
"$\impliedby$"
\newline
S - taietura arbitrara in G, e - muchie de cost minim in S, $T^*$ APM a.i. e $\notin$ E($T^*$). $T^*$ conex $\implies$ $\exists$ $e^*$ $\in$ E($T^*$) a.i. $e^* \in E(S)$.
\newline
Arborele T' = (V, E'), E' = E($T^*$) \textbackslash \{$e^*$\} $\cup$ \{e\}.
\newline
\[
\begin{rcases}
\begin{gathered}
c(T') = c(T^*) - c(e^*) + c(e)\\
c(e) = min(S) \leq c(e^*)
\end{gathered} \end{rcases} \Rightarrow \begin{gathered}
c(T') \leq c(T^*)
\end{gathered}
\]
$T^*$ este APM $\implies$ c(T') = c($T^*$), deci T' este un alt APM care il contine pe $\implies$ e = min(S), S taietura in G $\implies$ $\exists T^*$ APM, a.i. e $\in$ E($T^*$) $\Longleftrightarrow \exists S$ taietura in G, a.i. e are cost minim in S.
\newline\newline
b.
\newline
"$\implies$"
\newline
$\exists$ S taietura in G, e = uv muchie de cost minim in S. Daca $\nexists$ C circuit in G a.i. e $\in$ C $\implies$ e nu are cost maxim intr-un circuit. 
\newline
Daca $\exists$ $C_0, ..., C_k$ a.i. e $\in$ C (oricare circuit de la u la v), atunci $\exists e_1 \in C_1, ..., e_k \in C_k$ a.i. \{$e_1, ..., e_k$\} $\subset$ S.
\newline
e = min(S) $\implies$ c(e) \textless c($e_1$), ..., c(e) \textless c($e_k$) $\implies$ $\exists$ cel putin o muchie cu cost mai mare decat e $\implies$ e nu este muchie de cost maxim pe niciun circuit.
\newline\newline
"$\impliedby$"
\newline
Consideram e = uv, daca e nu este muchie de cost minim pe nicio taietura $\implies$ $\exists$ cel putin inca un drum de la u la v.
Fie $D_1, D_2, ..., D_k$ toate drumurile de la u la v. Multimea tuturor muchiilor din aceste drumuri genereaza o taietura in G. Intrucat e nu este de cost minim, c(e) este mai mare decat costurile tuturor celorlalte muchii din $D_1, D_2, ..., D_k$. Fiecare din aceste drumuri reunesc toate circuitele care il includ pe e, iar c(e) fiind mai mare decat a tuturor celorlalte muchii de pe circuite $\implies$ c(e) are cost maxim in cel putin un circuit. Daca e nu este de cost maxim pe vreun circuit $\implies$ e este de cost minim intr-o taietura.
\newline\newline
$\nexists T^*$ APM a.i. e $\in$ E($T^*$) $\Longleftrightarrow$ e este de cost maxim intr-un circuit
\newline\newline
c.
\newline
$\forall$ e - muchie aleasa arbitrar, aceasta este fie de cost minim intr-o taietura, fie de cost maxim intr-un circuit, iar asta presupune o colorare  in verde sau albastru, dar e este ales arbitrar $\implies$ nu se opreste cata vreme mai exista muchii rosii in graf care pot fi colorate fie in albastru, fie in verde.
\newline\newline
d.
\newline
La inceput toate muchiile vor fi colorate cu rosu. Alegand mereu o muchie colorata cu rosu, aceasta fie va fi colorata in albastru, fie in verde (rezulta din punctul anterior), deci numarul muchiilor colorate cu rosu va scadea la fiecare iteratie a buclei while. Cum multimea muchiilor este finita, algoritmul se va opri.
\newline\newline
La fiecare iteratie a buclei while exista o unica muchie e care sa indeplineasca conditia de cost minim in taietura si cost maxim in circuit, deci vom avea o unica succesiune de muchii care va constitui in final APM-ul H.

\end{proof}

\newpage

\begin{problema}{4}
\end{problema}
 
\begin{proof}[Solutie]:\newline\newline
a.
\newline
$T_i = T \setminus \{k_1, k_2, ..., k_{i-1}\}$
\newline
Eliminand oricare frunza din arbore, acesta ramane in continuare conex si aciclic $\forall$ i $\in$ \{1, 2, ..., p-1\}
\newline
$T_{p-1} = T \setminus \{k_1, k_2, ..., k_{p-2}\}$ = \{u,p\}, $T_{p-1}$ arbore $\implies$ up $\in$ E($T_{p-1}$), u, p, frunze. \newline
$k_{p-1}$ = min (u, p) = u $\implies$ $x_{p-1}$ = p $\forall$ T - construit arbitrar.
\newline\newline
b1.
\newline
$T_k$ = (V, E)
\newline
V($T_k$) = \{$i_k, ..., i_{p-1}\} \cup \{x_k, ..., x_{p-1}\}$
\newline
E($T_k$) = \{$i_k x_k, i_{k+1} x_{k+1}, ..., i_{p-1} x_{p-1}\}$
\newline
$\forall k_1, k_2, k_1 \leq k_2$
\newline
$i_{k_2}$ = min(V \textbackslash \{$i_1, ..., i_{k_1}, ..., i_{k_2 - 1}, x_{k_2}, ..., x_{p-1}$\} $\implies$ $i_{k_2}$ $\neq$ $i_{k_1}$ $\implies$ $i_k \notin$ \{$i_{k_1}, ..., i_{p-1}$\}
\newline
$i_{k}$ = min(V \textbackslash \{$i_1, ..., i_{k-1}, x_{k}, ..., x_{p-1}$\} $\implies$ $i_k \notin$ \{$x_{k}, ..., x_{p-1}$\}
\newline
$\implies$ $i_k$ face parte doar din muchia $i_k x_k$ $\implies$ $i_k$ este frunza in $T_k$
\newline\newline
b2.
\newline
V($T_k$) = \{$i_k, i_{k+1}, ..., i_{p-1}$\} $\cup$ \{$x_k, x_{k+1}, ..., x_{p-1}$\}
\newline
$i_k$ = min(V \textbackslash \{$i_1, ..., i_{k-1}, x_k, ..., x_{p-1}$\}) $\implies$ $i_k$ \textless\ u, $\forall$ u $\in$ \{$i_k, ..., i_{p-1}$\} $\setminus$ \{$x_k, ..., x_{p-1}$\} \textcircled{1}
\newline\newline
$x_i$ - arbitrar, i \textgreater\ k, daca $x_i$ = $x_j$, i $\neq$ j, j \textgreater\ k $\implies$ $x_i$ incident in $i_i x_i$ si in $i_j x_j$, $\implies$ $x_i$ nu este frunza.
\newline
$x_i$ $\neq$ $x_j$, $\forall$ j $\neq$ i, j \textgreater\ k $\implies$ $\exists$ j \textgreater\ i a.i. $x_i$ = min (V \textbackslash \{$i_1, ..., i_{j-1}, x_j, ..., x_{p-1}$\}) $\implies$ $x_i$ incident in $i_i x_i$ si in $i_j x_j$, deci $x_i$ nu este frunza.
\newline
$x_i$ - arbitrar $\implies$ $\nexists$ u frunza, a.i. u $\in$ \{$x_k, ..., x_{p-1}$\} \textcircled{2}
\newline\newline
\textcircled{1}, \textcircled{2} $\implies$ $i_k$ este cea mai mica frunza din $T_k$
\newline\newline
b3.
\newline
$T_k$ = T $\setminus$ \{$i_1, ..., i_{k-1}$\}
\newline
pct. b2 $\implies$ $i_k$ cea mai mica frunza din $T_k$
\newline
$i_k x_k \in$ E($T_k$)
\newline
Daca $x_k$ se afla pe pozitia k in emblema lui T $\forall$ k $\in$ \{1, ..., p-1\} $\implies$ x este emblema lui T.
\end{proof}

\end{document}
